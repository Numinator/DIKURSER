\documentclass[12pt, a4paper, hidelinks]{article}

% Packages:
\usepackage{graphicx}                   % For figure includes
\usepackage[T1]{fontenc}                % For mixing up \textsc{} with \textbf{}
\usepackage[utf8]{inputenc}             % For scandinavian input characters(æøå)
\usepackage{amsfonts, amsmath, amssymb} % For common mathsymbols and fonts
\usepackage[danish]{babel}              % For danish titles
\usepackage{hyperref}                   % For making links and refrences
\usepackage{url}                        % Just because {~_^}
\usepackage{array}                      % ...
\usepackage[usenames, dvipsnames, svgnames, table]{xcolor}
\usepackage{tabularx, colortbl}
\usepackage{verbatim} % For entering code snippets.
\usepackage{fancyvrb} % A "fancy" verbatim (for pseudo code).
\usepackage{listings} % For boxed codesnippets, and file includes. (begin)
\usepackage{lipsum}   % For generating dummy text at this demonstration
\usepackage{scrextend} % For den fede liste type

% Basic layout:
\setlength{\textwidth}{165mm}
\setlength{\textheight}{240mm}
\setlength{\parindent}{0mm}
\setlength{\parskip}{\parsep}
\setlength{\headheight}{0mm}
\setlength{\headsep}{0mm}
\setlength{\hoffset}{-2.5mm}
\setlength{\voffset}{0mm}
\setlength{\footskip}{15mm}
\setlength{\oddsidemargin}{0mm}
\setlength{\topmargin}{0mm}
\setlength{\evensidemargin}{0mm}

\newcolumntype{C}[1]{>{\centering\arraybackslash}p{#1}}

% Colors:
\definecolor{KU-red}{RGB}{144, 26, 30}

% Text Coloring:
\newcommand{\green}[1]{\textbf{\color{green}{#1}}}
\newcommand{\blue} [1]{\textbf{\color{blue} {#1}}}
\newcommand{\red}  [1]{\textbf{\color{red}  {#1}}}

% Simple Language Highlighting for F#
\definecolor{bluekeywords}{rgb}{0.13,0.13,1}
\definecolor{greencomments}{rgb}{0,0.5,0}
\definecolor{turqusnumbers}{rgb}{0.17,0.57,0.69}
\definecolor{redstrings}{rgb}{0.5,0,0}
\lstdefinelanguage{FSharp}
                  {morekeywords={let, new, match, with, rec, open,
                      module, namespace, type, of, member, and, for,
                      in, do, begin, end, fun, function, try, mutable,
                      if, then, else},
                    keywordstyle=\color{bluekeywords},
                    sensitive=false,
                    morecomment=[l][\color{greencomments}]{///},
                    morecomment=[l][\color{greencomments}]{//},
                    morecomment=[s][\color{greencomments}]{{(*}{*)}},
                    morestring=[b]",     stringstyle=\color{redstrings}
                  }
% You might want to change these lines at some point
\lstset{
  basicstyle=\ttfamily,
  columns=fullflexible,
  keepspaces=true,
  language=FSharp
}

\renewcommand\thesubsection{\thesection.\alph{subsection}}

% ************************* Start Document *****************
\begin{document}

% ************************* Page Header ********************
\begin{minipage}[b]{1.0\linewidth}
\includegraphics[height=30mm]{KULogo}

\vspace*{-16ex}
\begin{center}
    {\Large \bf DMA} \vspace*{1ex} \\
    {\large Gruppeopgave 3} \vspace*{1ex} \\
    {\large Frederik Kallestrup Mastratisi , Aiyu Liu og Rasmus Frydendal Kristensen}
\end{center}
\vspace*{-3pt}
{\color{KU-red}\hrule}
\end{minipage}
\vspace{2ex}

% **************** Assignment Starts Here ******************
\tableofcontents \newpage

\section{Størrelsesorden $n^2$}
Genrelt gør S8 + O8 at vi kun behøver at se på det største led


\begin{labeling}{$n^2$ + n log n: }
\item [\pmb{n + log n:}] Begge led er mindre strørrelsesorden end $n^2$ (se O2 og S3), derfor er hele udtrykket mindre størrelsesorden end $n^2$ (se O8 + S8)
\item [\pmb{$n^2 + 2^n$:}] $2^n$ er det største led og er større end $n^2$ (S5)
\item [\pmb{$n^2$ + n log n:}] log n er mindre end n og derfor er n log n mindre end $n^2$ (S2 + S7), det største led er $n^2$ som er O($n^2$).
\item [\pmb{$n^2(3+\sqrt{n})$:}] $n^2$ er ganget med et led som er afhængigt af n (i.e. ikke en konstant), og som ikke er aftagende. Derfor er dette udtryk større end $n^2$.
\item [\pmb{$(n+\sqrt(n))^2$:}] Dette er $n^2+n*\sqrt{n}+n$. Det største led er $n^2$, hvilket giver køretiden  O($n^2$) 
\end{labeling}

\section{Følgers værdier og størrelsesorden}

\subsection{} 
10, 10, 10: Hvert element er rekursivt defineret til at være lig med det forrige element, derfor den O(1)\vspace{1.3mm}\\
1, 5, 14: Med sumformlen kan man se at O($n^3$)\vspace{1.3mm}\\
0.1, 0.4, 0.9: Konstanter er ligegyldige (S6) O($n^2$)\vspace{1.3mm}\\
1.5, 2.25, 3.375: det har formmen $c^n$, $c > 0$ og hermed er O($a^n$)\vspace{1.3mm}\\

\subsection{}

$a_1 = 10, a_n = a_n-1 for n > 1$ O(1)

$b_n = \sum\limits_{k=1}^n k^2 $ O($n^2$)

$c_n = \frac{n^2}{10}$ O($n^3$)

$d_n = (\frac 3 2)^n$ O($1.5^n$)

\newpage

\section{Eksplicit udtryk}

$\sum\limits_{k=0}^n (2k+1)$

$2\cdot\sum\limits_{k=0}^n k +\sum\limits_{k=0}^n 1$

$\frac{2\cdot(n^2+n)}{2} + n + 1$

$n^2+2n+ 1$


% *********************** The End  *************************
\end{document}
