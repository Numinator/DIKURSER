\documentclass[12pt, a4paper, hidelinks]{article}

% Packages:
\usepackage{graphicx}                   % For figure includes
\usepackage[T1]{fontenc}                % For mixing up \textsc{} with \textbf{}
\usepackage[utf8]{inputenc}             % For scandinavian input characters(æøå)
\usepackage{amsfonts, amsmath, amssymb} % For common mathsymbols and fonts
\usepackage[danish]{babel}              % For danish titles
\usepackage{hyperref}                   % For making links and refrences
\usepackage{url}                        % Just because {~_^}
\usepackage{array}                      % ...
\usepackage[usenames, dvipsnames, svgnames, table]{xcolor}
\usepackage{tabularx, colortbl}
\usepackage{verbatim} % For entering code snippets.
\usepackage{fancyvrb} % A "fancy" verbatim (for pseudo code).
\usepackage{listings} % For boxed codesnippets, and file includes. (begin)
\usepackage{lipsum}   % For generating dummy text at this demonstration
\usepackage{scrextend} % For den fede liste type


% Basic layout:
\setlength{\textwidth}{165mm}
\setlength{\textheight}{240mm}
\setlength{\parindent}{0mm}
\setlength{\parskip}{\parsep}
\setlength{\headheight}{0mm}
\setlength{\headsep}{0mm}
\setlength{\hoffset}{-2.5mm}
\setlength{\voffset}{0mm}
\setlength{\footskip}{15mm}
\setlength{\oddsidemargin}{0mm}
\setlength{\topmargin}{0mm}
\setlength{\evensidemargin}{0mm}

\newcolumntype{C}[1]{>{\centering\arraybackslash}p{#1}}

% Colors:
\definecolor{KU-red}{RGB}{144, 26, 30}

% Text Coloring:
\newcommand{\green}[1]{\textbf{\color{green}{#1}}}
\newcommand{\blue} [1]{\textbf{\color{blue} {#1}}}
\newcommand{\red}  [1]{\textbf{\color{red}  {#1}}}

% Simple Language Highlighting for F#
\definecolor{bluekeywords}{rgb}{0.13,0.13,1}
\definecolor{greencomments}{rgb}{0,0.5,0}
\definecolor{turqusnumbers}{rgb}{0.17,0.57,0.69}
\definecolor{redstrings}{rgb}{0.5,0,0}
\lstdefinelanguage{FSharp}
                  {morekeywords={let, new, match, with, rec, open,
                      module, namespace, type, of, member, and, for,
                      in, do, begin, end, fun, function, try, mutable,
                      if, then, else},
                    keywordstyle=\color{bluekeywords},
                    sensitive=false,
                    morecomment=[l][\color{greencomments}]{///},
                    morecomment=[l][\color{greencomments}]{//},
                    morecomment=[s][\color{greencomments}]{{(*}{*)}},
                    morestring=[b]",     stringstyle=\color{redstrings}
                  }
% You might want to change these lines at some point
\lstset{
  basicstyle=\ttfamily,
  columns=fullflexible,
  keepspaces=true,
  language=FSharp
}

% ************************* Start Document *****************
\begin{document}

% ************************* Page Header ********************
\begin{minipage}[b]{1.0\linewidth}
\includegraphics[height=30mm]{KULogo}

\vspace*{-16ex}
\begin{center}
    {\Large \bf Diskret Matematik og Algoritmer 2016} \vspace*{1ex} \\
    {\large Ugeopgave 4} \vspace*{1ex} \\
    {\large Rasmus F, Aiyu L \& Frederik KM}
\end{center}
\vspace*{-3pt}
{\color{KU-red}\hrule}
\end{minipage}
\vspace{2ex}

% **************** Assignment Starts Here ******************
\tableofcontents \newpage

\setcounter{section}{0}
\setcounter{subsection}{-1}

\section{Del 1}

\subsection{(a)} 
\subsubsection{}
Den skal have: \\
container til dataen der skal oplærgers og 2 pointers som peger på det forrige og næste ellement eller til null \\

\subsubsection{}
\begin{lstlisting}
List::List (int z) : length (1) {
  head = new node(z)
  tail = head
}

\end{lstlisting}


\subsection{(b)}

Vi går ud fra at S er sorteret i ikke-faldene order \\
\begin{lstlisting}
void node::insertBefore (value v) {
  if (prev == null) {
    prev = new node(v)
    prev.next = &this
  } else {
    temp = new node(v)
    temp.prev = prev
    temp.next = &this
    prev.next = temp
    prev = temp    
  }

void node::insertAfter (value v) {
  if (next == null) {
    next = new node(v)
    next.prev = &this
  } else {
    temp = new node(v)
    temp.prev = &this
    temp.next = next 
    next.prev = temp
    next = temp    
  }

F(S, z) {
  i = S.head
  if (z < i.value) {i.insertBefore(z)}
  else 
    while (i != null && z < (i.next).value())
      i = i.next
    i.prev.insertAfter( z)
} 
\end{lstlisting}

\subsection{(c)}

Alle linjerne i   $F$ kører i konstant tid, med undtagelse af while-løkken. Den kører fra head til - i værst tænklige tilfælde - tails; altså over de $n$ elementer, og den gør det højest 1 gang. \
Altså er kørertiden $O(n)$.

\subsection{(d)}

Hver gang et tal bliver indsat, skal den går over elementerne i listen en gang: \\
\[ 1+2+...+ n - 1 + n = {n^2 + n \over 2}\]

Udtrykket oven over er $O(n^2)$

\section{Del 2}

\subsection{(a)}

En måde man kunne gører dette på er at gennemløbe S, og for hvert k'te element opretter man en knude som man hægter på B. Kørertiden ville således være $O(n)$.

Dette kræver dog at man kender længden på S. Enten ved at S er implementeret at længden er gemt i et medlem af S, ellers er man nødt til at gennemløbe S for at finde længden. Køretiden vil stadig være $O(n)$. 

\subsection{(b)}

Det ville tage $O(\sqrt{n})$, da dette er antallet af elementer i B. 

\subsection{(c)}




















% *********************** The End  *************************
\end{document}
